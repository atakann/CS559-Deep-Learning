\documentclass[10pt,journal,onecolumn]{IEEEtran}

\usepackage{cite}
\usepackage{amsmath,amssymb,amsfonts}
\usepackage{algorithmic}
\usepackage{graphicx}
\usepackage{textcomp}
\usepackage{xcolor}
\begin{document}
% The paper headers
\markboth{\today}
{}
% The only time the second header will appear is for the odd numbered pages
% after the title page when using the twoside option.
% 
% paper title
% Titles are generally capitalized except for words such as a, an, and, as,
% at, but, by, for, in, nor, of, on, or, the, to and up, which are usually
% not capitalized unless they are the first or last word of the title.
% Linebreaks \\ can be used within to get better formatting as desired.
% Do not put math or special symbols in the title.
\title{Fine-Grained Object Recognition in Remote Sensing Imagery}

%\author{Atakan Serbes \\
%Department of Computer Science\\
%Bilkent University, Bilkent, Ankara 06800, Turkey\\
%Email: atakan.serbes@bilkent.edu.tr}

\author{\IEEEauthorblockN{ Atakan Serbes, 21200694} \\
\IEEEauthorblockA{\textit{Department of Computer Science,} 
\textit{Bilkent University} \\
%Bilkent, Ankara 06800, Turkey \\
Email: atakan.serbes@bilkent.edu.tr}}

% make the title area
\maketitle
% As a general rule, do not put math, special symbols or citations
% in the abstract or keywords.
%\begin{abstract}
%The abstract goes here.
%\end{abstract}
%
%% Note that keywords are not normally used for peerreview papers.
%\begin{IEEEkeywords}
%IEEE, IEEEtran, journal, \LaTeX, paper, template.
%\end{IEEEkeywords}



\vspace{-10mm}
\section{Problem Description}
	Nowadays, the amount of data collected by the aerial means, especially by the satellites, is increasing day by day with the advance of technology. Remote sensing image classification is one of the fundamental tasks of image processing in this field. Deep Convolutional Neural Networks (DCNN) \cite{NIPS2012_4824} brought the state-of-the-art learning framework for the image recognition.
%\IEEEPARstart The problem
% You must have at least 2 lines in the paragraph with the drop letter
% (should never be an issue)

%\hfill mds
% 
%\hfill August 26, 2015
%
%\subsection{Subsection Heading Here}
%Subsection text here.
%

\section{Datasets}
Many datasets were examined for the project. And to study the problem of fine-grained object recognition and/or localization, three of the dataset collections were selected. The first of the three datasets are DOTA-v1.5 Large-Scale Dataset with 400,000 annotated instances with 16 categories which includes large areas like baseball diamonds, harbors as well as small parts such as small vehicles and helicopters. The second dataset considered is named PatternNet, a Large-Scale high resolution remote sensing dataset containing 38 different classes that varies from details like oil wells to areas like basketball courts. The third dataset may be the DLRSD dense labeling dataset


\section{Conclusion}
The conclusion goes here.

%\appendices
%\section{Proof of the First Zonklar Equation}
%Appendix one text goes here.

% you can choose not to have a title for an appendix
% if you want by leaving the argument blank
%\section{}
%Appendix two text goes here.


% use section* for acknowledgment
%\section*{Acknowledgment}
%The authors would like to thank...

% references section
%\begin{thebibliography}{1}

\bibliographystyle{IEEEtran}
\bibliography{referencess}

%\bibitem{IEEEhowto:kopka}
%H.~Kopka and P.~W. Daly, \emph{A Guide to \LaTeX}, 3rd~ed.\hskip 1em plus
 % 0.5em minus 0.4em\relax Harlow, England: Addison-Wesley, 1999.

%\end{thebibliography}

\end{document}